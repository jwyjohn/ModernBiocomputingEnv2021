% Session01.tex
\section{Shell基础和正则表达式}

\subsection{操作环境简介}
笔者使用的软硬件环境如下所示,其他现代的Linux发行版一般不经过特殊的配置也可以使用。关于如何安装Linux系统,请自行查阅相关资料。
\begin{lstlisting}[frame=none]
OS: Manjaro Linux x86_64 
Host: SuperMicro Server
Kernel: 5.4.95-1-MANJARO 
Shell: bash 5.1.0 
Terminal: /dev/pts/0 
CPU: 06/55 Engineering Sample (56) @ 3.000GHz 
Memory: 1714MiB / 128573MiB 
\end{lstlisting}

对于使用苹果M1芯片或者其他Arm芯片的机器,可能会存在兼容性问题,因而建议在x86机器上进行实验。

\subsection{Bash基本操作}
打开一个shell后,一般会出现这样的提示符:
\begin{lstlisting}[frame=none]
[john@manjaro ~]$ 
\end{lstlisting}
提示符分为几个部分,一般来说,会显示用户名,主机名,工作目录以及一个\texttt{\$}或\texttt{\#}。若显示为\texttt{\$},则当前用户为普通用户,而\texttt{\#}则代表当前用户为特权用户root。

在shell中输入命令,并且按回车键执行。例子如下:
\begin{lstlisting}[frame=none]
[john@manjaro ~]$ echo $SHELL
/bin/bash
[john@manjaro ~]$ pwd
/home/john
[john@manjaro ~]$ ls -hl
total 12K
drwxr-xr-x 3 john john 4.0K Feb 21 00:33 Downloads
drwxr-xr-x 3 john john 4.0K Feb 23 15:52 raid0
drwxr-xr-x 7 john john 4.0K Feb 25 10:53 SSD
[john@manjaro ~]$ 
\end{lstlisting}

使用键盘上的左右键可以移动光标,上下键可以查询到历史命令。使用\texttt{Tab}键可以自动补全命令(不十分智能),使用\texttt{Ctrl-R}键可以搜索历史指令(不十分准确)。

对于bash而言,在输入较长但命令时,如果需要修改命令的开头或者结尾,有一些快捷键会比直接使用方向键更为方便。例如:
\begin{itemize}
  \item \texttt{Ctrl-A} 回行首
  \item \texttt{Ctrl-E} 回行尾
  \item \texttt{Alt-B} 向前移动一个单词(Alt有关的快捷键在Mac上兼容性不佳)
  \item \texttt{Alt-F} 向后移动一个单词
  \item \texttt{Ctrl-U} 删除光标前的所有内容
  \item \texttt{Ctrl-K} 删除光标后的所有内容
  \item \texttt{Ctrl-W} 删除光标前的一个单词
  \item \texttt{Alt-D} 删除光标后的一个单词
  \item \texttt{Ctrl-\_} 撤销上一次操作
\end{itemize}
事实上,对于使用了\texttt{GNU Readline}库的程序(包括大部分shell),上述的这些快捷键大多是适用的。

此外,Bash中还有管道和重定向等操作,这里不再赘述,建议参考扩展阅读中的内容。

\subsection{Linux下常用工具简介}
Bash中可以使用的命令有两种,一种是系统中的可执行程序,另一种是Bash内置的指令。对于系统中的可执行程序,可以使用\texttt{whereis}指令来查找其位置,例如:
\begin{lstlisting}[frame=none]
[john@manjaro ~]$ whereis pwd
pwd: /usr/bin/pwd /usr/include/pwd.h /usr/share/man/man1/pwd.1p.gz /usr/share/man/man1/pwd.1.gz
\end{lstlisting}
而对于内置命令,如\texttt{cd}等,则不存在与之对应等可执行文件,例如:
\begin{lstlisting}[frame=none]
[john@manjaro ~]$ whereis cd
cd: /usr/share/man/man1/cd.1p.gz
\end{lstlisting}
想要查看系统中等所有指令,可以使用\texttt{compgen}这个内置命令,其后的参数有\texttt{-cabkA},功能不同,例如:
\begin{lstlisting}[frame=none]
[john@manjaro ~]$ compgen -c
cp
df
egrep
fgrep
free
grep
ls
more
np
...
\end{lstlisting}
经常会使用的工具被一些同学编写成了一个rap。
\begin{itemize}
  \item \textit{A是awk,跑起来慢如蜗牛,}
  \item \textit{B是biff,自动查收新电邮。}
  \item \textit{C是cc,地球黑客都知道,}
  \item \textit{D是dd,什么文件全能搞。}
  \item \textit{E是emacs,支持许多组合键,}
  \item \textit{F是fsck,将B树进行重建。}
  \item \textit{G是grep,堪称代码名侦探,}
  \item \textit{H是halt,注意电源不会断。}
  \item \textit{I是indent,时不时引发圣战,}
  \item \textit{J是join,简直毫无存在感。}
  \item \textit{K是kill,你也能做个boss,}
  \item \textit{L是lex,可惜不兼容DOS。}
  \item \textit{M是more,后被less取代,}
  \item \textit{N是nice,但没有那么和蔼。}
  \item \textit{0是od,二进制阅读方便,}
  \item \textit{P是passwd,字符串读取两遍。}
  \item \textit{Q是quota,伯克利系统特色,}
  \item \textit{R是reboot,重启操作系统。}
  \item \textit{S是ssh,连接远程电脑,}
  \item \textit{T是tar,压缩包打包好。}
  \item \textit{∪是uniq,sort之后再整理,}
  \item \textit{V是vi,退出程序不容易。}
  \item \textit{W是whoami,立刻报上你大名,}
  \item \textit{X .....就是X,历史上不少批评。}
  \item \textit{Y是yes,唯一印象会刷屏,}
  \item \textit{Z是zcat,解压文本它能行。}
\end{itemize}
关于这些工具的用途和历史,有兴趣可以网上查阅资料,或者使用\texttt{man}指令查看系统中的手册。

\subsection{正则表达式}
正则表达式(regex)是用于匹配字符串的一个pattern。很多程序都不同程度地支持正则表达式,如\texttt{vim},\texttt{emacs},\texttt{grep},\texttt{sed},\texttt{awk}以及各类编译器解释器等。下面以\texttt{grep}系列为例介绍正则表达式的使用。注意,使用正则表达式时,最好将pattern用引号包住,以免被shell转义。

首先,生成我们用于演示正则表达式的文件:
\begin{lstlisting}[frame=none]
[john@manjaro Session01]$ compgen -c > foobar.txt
[john@manjaro Session01]$ head foobar.txt 
cp
df
egrep
fgrep
free
grep
hg
ll
ls
more
\end{lstlisting}

\subsubsection{直接匹配}
一个常规的字符串也是正则表达式,例如:
\begin{lstlisting}[frame=none]
[john@manjaro Session01]$ grep rep foobar.txt 
egrep
fgrep
grep
gawklibpath_prepend
gawkpath_prepend
zipgrep
pgrep
...
\end{lstlisting}
\subsubsection{通配符}
在正则表达式中,字符"\texttt{.}"可以匹配任意的字符,例如:
\begin{lstlisting}[frame=none]
[john@manjaro Session01]$ grep .g.ep foobar.txt 
egrep
fgrep
zipgrep
pgrep
xzegrep
...
\end{lstlisting}
除了字符"\texttt{.}"外,被"\texttt{[ ]}"括起来的字符也是通配符,例如:
\begin{lstlisting}[frame=none]
[john@manjaro Session01]$ grep [ef]grep foobar.txt 
egrep
fgrep
xzegrep
...
\end{lstlisting}
若把一些字符用"\texttt{[\^{} ]}"括起来,则表示匹配不含这些字符的pattern,例如:
\begin{lstlisting}[frame=none]
[john@manjaro Session01]$ grep [^ef]grep foobar.txt 
zipgrep
pgrep
lzgrep
xzgrep
zgrep
...
\end{lstlisting}
使用"\texttt{[ ]}"的方式比较灵活多变,比如:
\begin{lstlisting}[frame=none]
[john@manjaro Session01]$ grep [a-zA-Z]grep foobar.txt 
egrep
fgrep
zipgrep
pgrep
xzegrep
...
[john@manjaro Session01]$ grep [[:alpha:]]grep foobar.txt 
egrep
fgrep
zipgrep
pgrep
xzegrep
...
\end{lstlisting}
此外还有\texttt{[:lower:]},\texttt{[:upper:]},\texttt{[:digit:]},\texttt{[:alnum:]},\texttt{[:punct:]},\texttt{[:space:]}等。

\subsubsection{定位行首行尾}
如果要在行首或行尾进行匹配,可以使用"\texttt{\^{}}"和"\texttt{\$}",例如:
\begin{lstlisting}[frame=none]
[john@manjaro Session01]$ grep ^[a-z]grep foobar.txt 
egrep
fgrep
pgrep
egrep
zgrep
fgrep
[john@manjaro Session01]$ grep pc.$ foobar.txt 
setpci
lspci
ipcs
grpck
dhcpcd
\end{lstlisting}
注意与上面的"\texttt{[\^{} ]}"相区别。
\subsubsection{重复字符}
若要匹配若干重复的字符,可以使用"\texttt{*}",例如:
\begin{lstlisting}[frame=none]
[john@manjaro Session01]$ grep ^l*s foobar.txt 
ls
select
set
shift
shopt
...
\end{lstlisting}
注意,0次也会被匹配到。也可以利用"\texttt{\{n,m\}}"指定重复字符的次数,例如:
\begin{lstlisting}[frame=none]
[john@manjaro Session01]$ egrep ^l\{1,\} foobar.txt 
ll
ls
let
local
logout
...
\end{lstlisting}
\subsubsection{子表达式}
将表达式用"\texttt{()}"括起来,程序就将其当作一个字符处理,例如:
\begin{lstlisting}[frame=none]
[john@manjaro Session01]$ egrep ^\(ls\)\{1,\} foobar.txt 
ls
lslocks
lsns
lsoptdepends
...
lscp
lsb_release
lslslslslsls
\end{lstlisting}
\subsubsection{回溯引用}
回溯引用是指模式的后半部分引用在前半部分中定义的子表达式。先准备测试用的素材:
\begin{lstlisting}[frame=none]
[john@manjaro Session01]$ cat <<EOF>>foobar.txt 
> <body>
<h1>Welcome to my page</H1>
Content is divided into twosections:<br>
<h2>Introduction</h2>
Information about me.
<H2>Hobby</H2>
Information about my hobby.
<h2>This is invalid HTML</h3>
</body>
> EOF
\end{lstlisting}
如果不使用回溯引用,例如:
\begin{lstlisting}[frame=none]
[john@manjaro Session01]$ egrep '<[hH][1-6]>.*?</[hH][1-6]>' foobar.txt 
<h1>Welcome to my page</H1>
<h2>Introduction</h2>
<H2>Hobby</H2>
<h2>This is invalid HTML</h3>
\end{lstlisting}
注意到最后一行的匹配两个标签不相符合,而使用"\texttt{$\backslash$n}"来代替上n次被匹配到的字符串,例如:
\begin{lstlisting}[frame=none]
[john@manjaro Session01]$ egrep '<[hH]([1-6])>.*?</[hH]\1>' foobar.txt 
<h1>Welcome to my page</H1>
<h2>Introduction</h2>
<H2>Hobby</H2>
\end{lstlisting}
则可以达到预期的效果。
\subsubsection{扩展的正则表达式}
将表达式用"\texttt{|}"连接,表示匹配前一个表达式或后一个表达式,例如:
\begin{lstlisting}[frame=none]
[john@manjaro Session01]$ egrep '^(ls|su).*' foobar.txt 
ls
suspend
lslocks
lsns
lsoptdepends
ls
sulogin
lsinitcpio
...
\end{lstlisting}
一些符号在扩展的正则表达式中可以作为重复次数的缩写,比如"\texttt{*}"表示比配0到无穷次,"\texttt{+}"1到无穷次,以及"\texttt{?}"表示匹配0到1次,例如:
\begin{lstlisting}[frame=none]
[john@manjaro Session01]$ egrep ls.+ foobar.txt 
else
false
lslocks
lsns
lsoptdepends
gnutls-cli
nl-cls-add
...
\end{lstlisting}
\subsubsection{非正常字符}
有时会匹配一些非正常字符,例如:
\begin{lstlisting}[frame=none]
\cx	匹配由x指明的控制字符。例如, \cM 匹配一个 Control-M 或回车符。x 的值必须为 A-Z 或 a-z 之一。否则,将 c 视为一个原义的 'c' 字符。
\f	匹配一个换页符。等价于 \x0c 和 \cL。
\n	匹配一个换行符。等价于 \x0a 和 \cJ。
\r	匹配一个回车符。等价于 \x0d 和 \cM。
\s	匹配任何空白字符,包括空格、制表符、换页符等等。等价于 [ \f\n\r\t\v]。注意 Unicode 正则表达式会匹配全角空格符。
\S	匹配任何非空白字符。等价于 [^ \f\n\r\t\v]。
\t	匹配一个制表符。等价于 \x09 和 \cI。
\v	匹配一个垂直制表符。等价于 \x0b 和 \cK。
\w  等价于[A-Za-z0-9_].
\W  等价于[^A-Za-z0-9_].
\b	匹配一个单词边界,即字与空格间的位置。
\B	非单词边界匹配。
\end{lstlisting}
\subsubsection{用正则表达式处理fasta文件}
首先,准备实验用的文件:
\begin{lstlisting}[frame=none]
[john@manjaro Session01]$ wget https://ftp.ncbi.nlm.nih.gov/genomes/refseq/bacteria/Escherichia_coli/reference/GCF_000005845.2_ASM584v2/GCF_000005845.2_ASM584v2_cds_from_genomic.fna.gz && gzip -d GCF_000005845.2_ASM584v2_cds_from_genomic.fna.gz
\end{lstlisting}
浏览文件的开头,对文件作一个大致的了解:
\begin{lstlisting}[frame=none]
[john@manjaro Session01]$ head GCF_000005845.2_ASM584v2_cds_from_genomic.fna 
>lcl|NC_000913.3_cds_NP_414542.1_1 [gene=thrL] [locus_tag=b0001] [db_xref=UniProtKB/Swiss-Prot:P0AD86] [protein=thr operon leader peptide] [protein_id=NP_414542.1] [location=190..255] [gbkey=CDS]
ATGAAACGCATTAGCACCACCATTACCACCACCATCACCATTACCACAGGTAACG...
\end{lstlisting}
统计序列个数,将"\texttt{grep}"与"\texttt{wc}"连用,在这个例子中,就是大肠杆菌中CDS的数量:
\begin{lstlisting}[frame=none]
[john@manjaro Session01]$ grep "^>" GCF_000005845.2_ASM584v2_cds_from_genomic.fna |wc -l
4357
\end{lstlisting}
统计核酸总数,同样将"\texttt{grep}"与"\texttt{wc}"连用,注意到扣除换行符,这里使用了多数linux发行版自带的计算器"\texttt{bc}":
\begin{lstlisting}[frame=none]
[john@manjaro Session01]$ echo \
$(grep "^[^>]" GCF_000005845.2_ASM584v2_cds_from_genomic.fna |wc -c) - \
$(grep "^[^>]" GCF_000005845.2_ASM584v2_cds_from_genomic.fna |wc -l)|bc
4091782
\end{lstlisting}
或者使用"\texttt{grep -o}",例如:
\begin{lstlisting}[frame=none]
[john@manjaro Session01]$ grep "^[^>]" GCF_000005845.2_ASM584v2_cds_from_genomic.fna |grep -o "[ATCG]"| wc -l
4091782
\end{lstlisting}
例如我们要寻找其中对PAM序列(NGG,CRISPR/CAS9操作必备),则可以进行如下操作:
\begin{lstlisting}[frame=none]
[john@manjaro Session01]$ grep -b "[ATCG]GG" GCF_000005845.2_ASM584v2_cds_from_genomic.fna
\end{lstlisting}

\subsection*{扩展阅读}
\begin{itemize}
  \item 你所不知的 GNU Readline | Linux 中国 - 知乎\\
  \url{https://zhuanlan.zhihu.com/p/264772139}
  \item 正则表达式 – 语法 | 菜鸟教程 - RUNOOB.COM\\
  \url{https://www.runoob.com/regexp/regexp-syntax.html}
  \item Vim入门基础 - 简书\\
  \url{https://www.jianshu.com/p/bcbe916f97e1}
  \item linux shell 管道命令(pipe)使用及与shell重定向区别\\
  \url{https://www.cnblogs.com/chengmo/archive/2010/10/21/1856577.html}
\end{itemize}
\begin{lstlisting}[frame=none]

\end{lstlisting}
